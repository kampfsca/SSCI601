% BEAMER ----
% This is just here so I know exactly what I'm looking at in Rstudio when messing with stuff.
\documentclass[10pt,ignorenonframetext,,aspectratio=169]{beamer}
\usefonttheme{serif}
\setbeamertemplate{caption}[numbered]
\setbeamertemplate{caption label separator}{: }
\setbeamercolor{caption name}{fg=normal text.fg}
\usepackage{lmodern}
\usepackage{amssymb,amsmath}
\usepackage{ifxetex,ifluatex}
\usepackage{fixltx2e} % provides \textsubscript
\ifnum 0\ifxetex 1\fi\ifluatex 1\fi=0 % if pdftex
  \usepackage[T1]{fontenc}
  \usepackage[utf8]{inputenc}
\else % if luatex or xelatex
  \ifxetex
    \usepackage{mathspec}
  \else
    \usepackage{fontspec}
  \fi
  \defaultfontfeatures{Ligatures=TeX,Scale=MatchLowercase}
  \newcommand{\euro}{€}
\fi
% use upquote if available, for straight quotes in verbatim environments
\IfFileExists{upquote.sty}{\usepackage{upquote}}{}
% use microtype if available
\IfFileExists{microtype.sty}{%
\usepackage{microtype}
\UseMicrotypeSet[protrusion]{basicmath} % disable protrusion for tt fonts
}{}
\usepackage{graphicx,grffile}
\makeatletter
\def\maxwidth{\ifdim\Gin@nat@width>\linewidth\linewidth\else\Gin@nat@width\fi}
\def\maxheight{\ifdim\Gin@nat@height>\textheight0.8\textheight\else\Gin@nat@height\fi}
\makeatother
% Scale images if necessary, so that they will not overflow the page
% margins by default, and it is still possible to overwrite the defaults
% using explicit options in \includegraphics[width, height, ...]{}
\setkeys{Gin}{width=\maxwidth,height=\maxheight,keepaspectratio}

% Comment these out if you don't want a slide with just the
% part/section/subsection/subsubsection title:
\AtBeginPart{
  \let\insertpartnumber\relax
  \let\partname\relax
  \frame{\partpage}
}
\AtBeginSection{
  \let\insertsectionnumber\relax
  \let\sectionname\relax
  \frame{\sectionpage}
}
\AtBeginSubsection{
  \let\insertsubsectionnumber\relax
  \let\subsectionname\relax
  \frame{\subsectionpage}
}

\setlength{\emergencystretch}{3em}  % prevent overfull lines
\providecommand{\tightlist}{%
  \setlength{\itemsep}{0pt}\setlength{\parskip}{0pt}}
\setcounter{secnumdepth}{0}

\title{Research on neighborhood effects on health in the United States:
A systematic review of study characteristics}
\subtitle{\textit{Arcaya et al (2016)}}
\author{Andy Kampfschulte}
\date{May 25, 2022}


%% Here's everything I added.
%%--------------------------

\usepackage{graphicx}
\usepackage{rotating}
\setbeamertemplate{caption}[numbered]
\usepackage{hyperref}
\usepackage{caption}
\usepackage[normalem]{ulem}
%\mode<presentation>
\usepackage{wasysym}
\usepackage {textpos}
\usepackage{tikz}
%\usepackage{amsmath}


% Get rid of navigation symbols.
%-------------------------------
\setbeamertemplate{navigation symbols}{}

% Optional institute tags and titlegraphic.
% Do feel free to change the titlegraphic if you don't want it as a Markdown field.
%----------------------------------------------------------------------------------
%

% \titlegraphic{\includegraphics[width=0.3\paperwidth]{\string~/Dropbox/teaching/clemson-academic.png}} % <-- if you want to know what this looks like without it as a Markdown field.
% -----------------------------------------------------------------------------------------------------



% Some additional title page adjustments.
%----------------------------------------
\setbeamertemplate{title page}[empty]
%\date{}
\setbeamerfont{subtitle}{size=\small}

\setbeamercovered{transparent}

\pgfdeclareimage [height=.25\paperheight] {SSI} {figures/SSI}
\pgfdeclareimage [height=0.6cm] {USC} {figures/USC}
\pgfdeclareimage [height=2.6cm] {Seal} {figures/Seal}
\pgfdeclareimage [width = \paperwidth] {SSCI} {figures/SSCI}
\pgfdeclareimage [width = \paperwidth] {PHP} {figures/SSCIPHP}

\institute { \pgfuseimage {SSI}  }


% Some optional colors. Change or add as you see fit.
%---------------------------------------------------
\definecolor{USCred}{HTML}{990000}
\definecolor{USCyellow}{HTML}{FFCC00}
\definecolor{altUSCred}{HTML}{a30a35}
\definecolor{altUSCyellow}{HTML}{febe10}





% Some optional color adjustments to Beamer. Change as you see fit.
%------------------------------------------------------------------
\setbeamercolor{frametitle}{fg=USCred,bg=white}
\setbeamercolor{title}{fg=USCred,,bg=white}
\setbeamercolor{local structure}{fg=USCred,}
\setbeamercolor{section in toc}{fg=USCred,bg=white}
% \setbeamercolor{subsection in toc}{fg=USCyellow,bg=white}
\setbeamercolor{footline}{fg=altUSCred!50, bg=white}
\setbeamercolor{block title}{fg=USCyellow,bg=white}


\let\Tiny=\tiny


% Sections and subsections should not get their own damn slide.
%--------------------------------------------------------------
\AtBeginPart{}
\AtBeginSection{}
\AtBeginSubsection{}
\AtBeginSubsubsection{}

% Suppress some of Markdown's weird default vertical spacing.
%------------------------------------------------------------
\setlength{\emergencystretch}{0em}  % prevent overfull lines
\setlength{\parskip}{0pt}


% Allow for those simple two-tone footlines I like.
% Edit the colors as you see fit.
%--------------------------------------------------
%\defbeamertemplate*{footline}{my footline}{%
%    \ifnum\insertpagenumber=1
%    \hbox{%
%        \begin{beamercolorbox}[wd=\paperwidth,ht=.8ex,dp=1ex,center]{}%
%      % empty environment to raise height
%        \end{beamercolorbox}%
%    }%
%    \vskip0pt%
%    \else%
%        \Tiny{
%            \hfill
%		\vspace*{1pt}%
 %           \insertframenumber/\inserttotalframenumber \hspace*{0.1cm}%
  %          \newline%
 %           \color{USCred}{\rule{\paperwidth}{0.4mm}}\newline%
 %           \color{USCyellow}{\rule{\paperwidth}{.4mm}}%
 %       }%
 %   \fi%
%}



% Trying some Footer Adjustments
%----------------------------------------
\setbeamercolor{section in head/foot}{fg=white, bg=altUSCred}

\defbeamertemplate*{footline}{my footline}{%
  \begin{beamercolorbox}[wd=\paperwidth,ht=8.25ex,dp=1ex,center]{section in head/foot}%
    \usebeamerfont{author in head/foot} \pgfuseimage {SSCI}
	\setlength{\tabcolsep}{0pt}
  \end{beamercolorbox}%

  \leavevmode%
  \hbox{%
  \begin{beamercolorbox}[wd=.333333\paperwidth,ht=2.0ex,dp=1ex,left]{section in head/foot}%
    \usebeamerfont{author in head/foot}\hspace{2ex}\insertshortauthor
  \end{beamercolorbox}%
  \begin{beamercolorbox}[wd=.333333\paperwidth,ht=2.0ex,dp=1ex,center]{section in head/foot}%
    \usebeamerfont{title in head/foot}\insertshorttitle
  \end{beamercolorbox}%
  \begin{beamercolorbox}[wd=.333333\paperwidth,ht=2.0ex,dp=1ex,right]{section in head/foot}%
    \usebeamerfont{date in head/foot}\insertshortdate{}\hspace*{2em}
    \insertframenumber{} / \inserttotalframenumber\hspace*{2ex}
  \end{beamercolorbox}}%

  \vskip0pt%
}

\usebackgroundtemplate{%
\tikz[remember picture,overlay] %
	\node[opacity=0.1, anchor = south east] at ([yshift=.14\paperheight] current page.south east)%
	{\pgfuseimage {Seal}};}

% Various cosmetic things, though I must confess I forget what exactly these do and why I included them.
%-------------------------------------------------------------------------------------------------------
\setbeamercolor{structure}{fg=blue}
\setbeamercolor{local structure}{parent=structure}
\setbeamercolor{item projected}{parent=item,use=item,fg=USCred,bg=white}
\setbeamercolor{enumerate item}{parent=item}

% Adjust some item elements. More cosmetic things.
%-------------------------------------------------
\setbeamertemplate{itemize item}{\color{USCred}$\bullet$}
\setbeamertemplate{itemize subitem}{\color{USCred}\scriptsize{$\bullet$}}
\setbeamertemplate{itemize/enumerate body end}{\vspace{.6\baselineskip}} % So I'm less inclined to use \medskip and \bigskip in Markdown.


% Automatically center images
% ---------------------------
% Note: this is for ![](image.png) images
% Use "fig.align = "center" for R chunks

\usepackage{etoolbox}

\AtBeginDocument{%
  \letcs\oig{@orig\string\includegraphics}%
  \renewcommand<>\includegraphics[2][]{%
    \only#3{%
      {\centering\oig[{#1}]{#2}\par}%
    }%
  }%
}

% I think I've moved to xelatex now. Here's some stuff for that.
% --------------------------------------------------------------
% I could customize/generalize this more but the truth is it works for my circumstances.

\ifxetex
\setbeamerfont{title}{family=\fontspec{serif}}
\setbeamerfont{frametitle}{family=\fontspec{serif}}
\usepackage[font=small,skip=0pt]{caption}
 \else
 \fi

% Some random stuff now...
% ------------------------

\usepackage{tikz}

\newcommand{\shrug}[1][]{%
\begin{tikzpicture}[baseline,x=0.8\ht\strutbox,y=0.8\ht\strutbox,line width=0.125ex,#1]
\def\arm{(-2.5,0.95) to (-2,0.95) (-1.9,1) to (-1.5,0) (-1.35,0) to (-0.8,0)};
\draw \arm;
\draw[xscale=-1] \arm;
\def\headpart{(0.6,0) arc[start angle=-40, end angle=40,x radius=0.6,y radius=0.8]};
\draw \headpart;
\draw[xscale=-1] \headpart;
\def\eye{(-0.075,0.15) .. controls (0.02,0) .. (0.075,-0.15)};
\draw[shift={(-0.3,0.8)}] \eye;
\draw[shift={(0,0.85)}] \eye;
% draw mouth
\draw (-0.1,0.2) to [out=15,in=-100] (0.4,0.95);
\end{tikzpicture}}

% header includes go last.

\newenvironment{cols}[1][]{}{}

\newenvironment{col}[1]{\begin{minipage}{#1}\ignorespaces}{%
\end{minipage}
\ifhmode\unskip\fi
\aftergroup\useignorespacesandallpars}

\def\useignorespacesandallpars#1\ignorespaces\fi{%
#1\fi\ignorespacesandallpars}

\makeatletter
\def\ignorespacesandallpars{%
  \@ifnextchar\par
    {\expandafter\ignorespacesandallpars\@gobble}%
    {}%
}
\makeatother
\usepackage{forest}
\usepackage{amsmath}
\usepackage{booktabs}
\usepackage{multicol}
\usepackage{makecell}
\usepackage{bm}
\usepackage{upgreek}

% Okay, and begin the actual document...

\begin{document}
\frame{\titlepage}

\hypertarget{central-themes}{%
\section{Central Themes}\label{central-themes}}

\hypertarget{primary-objectives}{%
\subsection{Primary Objectives}\label{primary-objectives}}

\begin{frame}{Primary Objectives}
\begin{itemize}
\tightlist
\item
  \emph{Provide new data on the characteristics of a broad set of
  neighborhoods and health studies over the past 20 years as a resource
  to better un-derstand the state of the ``neighborhood effects on
  health'' science}
\item
  \emph{Reflect on previous agendas to advance neighborhoods and health
  research, highlighting goals that have not yet been met by the
  existing literature}
\end{itemize}
\end{frame}

\hypertarget{methods}{%
\section{Methods}\label{methods}}

\begin{frame}{Methods}
\includegraphics[width=\textwidth,height=0.8\textheight]{C:/Users/andyk/Documents/Projects/SSCI601/Arcaya2016/fig1.png}
\end{frame}

\hypertarget{central-themes-1}{%
\subsection{Central Themes}\label{central-themes-1}}

\begin{frame}{Central Themes}
\begin{cols}

\begin{col}{0.6\textwidth}

\begin{itemize}
\tightlist
\item
  Cross-Sectional studies are abundant

  \begin{itemize}
  \tightlist
  \item
    Other designs are much needed to address confounding
  \end{itemize}
\item
  Census Geographies abound

  \begin{itemize}
  \tightlist
  \item
    Are study hypotheses truly \emph{a-priori}
  \item
    Convenience sampling
  \end{itemize}
\item
  GIS-related topics are often overlooked, as other commonplace
  statistics

  \begin{itemize}
  \tightlist
  \item
    MAUP
  \item
    3+ Level Modelling
  \item
    Spatial Relationships
  \end{itemize}
\item
  Concentration on just a few topics
\end{itemize}

\end{col}

\begin{col}{0.4\textwidth}
\includegraphics{C:/Users/andyk/Documents/Projects/SSCI601/Arcaya2016/table2.png}

\end{col}

\end{cols}
\end{frame}

\hypertarget{central-themes-2}{%
\subsection{Central Themes}\label{central-themes-2}}

\begin{frame}{Central Themes}
\begin{cols}

\begin{col}{0.6\textwidth}

\begin{itemize}
\tightlist
\item
  Cross-Sectional studies are abundant

  \begin{itemize}
  \tightlist
  \item
    Other designs are much needed to address confounding
  \end{itemize}
\item
  Census Geographies abound

  \begin{itemize}
  \tightlist
  \item
    Are study hypotheses truly \emph{a-priori}
  \item
    Convenience sampling
  \end{itemize}
\item
  GIS-related topics are often overlooked, as other commonplace
  statistics

  \begin{itemize}
  \tightlist
  \item
    MAUP
  \item
    3+ Level Modelling
  \item
    Spatial Relationships
  \end{itemize}
\item
  Concentration on just a few topics
\end{itemize}

\end{col}

\begin{col}{0.4\textwidth}
\includegraphics{C:/Users/andyk/Documents/Projects/SSCI601/Arcaya2016/table3.png}

\end{col}

\end{cols}
\end{frame}

\hypertarget{central-themes-3}{%
\subsection{Central Themes}\label{central-themes-3}}

\begin{frame}{Central Themes}
\begin{cols}

\begin{col}{0.6\textwidth}

\begin{itemize}
\tightlist
\item
  Cross-Sectional studies are abundant

  \begin{itemize}
  \tightlist
  \item
    Other designs are much needed to address confounding
  \end{itemize}
\item
  Census Geographies abound

  \begin{itemize}
  \tightlist
  \item
    Are study hypotheses truly \emph{a-priori}
  \item
    Convenience sampling
  \end{itemize}
\item
  GIS-related topics are often overlooked, as other commonplace
  statistics

  \begin{itemize}
  \tightlist
  \item
    MAUP
  \item
    3+ Level Modelling
  \item
    Spatial Relationships
  \end{itemize}
\item
  Concentration on just a few topics
\end{itemize}

\end{col}

\begin{col}{0.4\textwidth}
\includegraphics{C:/Users/andyk/Documents/Projects/SSCI601/Arcaya2016/table4.png}

\end{col}

\end{cols}
\end{frame}

\hypertarget{interdisciplinary-connections-evaluation}{%
\section{Interdisciplinary Connections \&
Evaluation}\label{interdisciplinary-connections-evaluation}}

\hypertarget{interdisciplinary-connections-evaluation-1}{%
\subsection{Interdisciplinary Connections \&
Evaluation}\label{interdisciplinary-connections-evaluation-1}}

\begin{frame}{Interdisciplinary Connections \& Evaluation}
\textbf{Interdisciplinary Connections}

\begin{itemize}
\tightlist
\item
  Incorporating Neighborhood elements connects people to space

  \begin{itemize}
  \tightlist
  \item
    GIS elements, Demography, Public Health Epidemiology
  \end{itemize}
\end{itemize}

\textbf{Evaluation}

\begin{itemize}
\tightlist
\item
  Limited to multi-level modelling
\item
  A good review of the current state of the literature
\item
  Identifies common themes, areas for future progress
\end{itemize}
\end{frame}

\hypertarget{questions}{%
\section{Questions}\label{questions}}

\hypertarget{questions-1}{%
\subsection{Questions}\label{questions-1}}

\begin{frame}{Questions}
\begin{itemize}
\tightlist
\item
  What are your thoughts on the Cross-Sectional Observational study of
  neighborhood effects? Is it completely overplayed at this point?
\item
  What are some other approaches to compliment/contrast/improve
  multi-level modelling with neighborhood effects?
\item
  Is this field of work getting more technical and deductive?
\end{itemize}
\end{frame}


\section[]{}
\frame{\small \frametitle{Table of Contents}
\tableofcontents}
\end{document}
